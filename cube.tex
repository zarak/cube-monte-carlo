%%%%%%%%%%%%%%%%%%%%%%%%%%%%% Define Article %%%%%%%%%%%%%%%%%%%%%%%%%%%%%%%%%%
\documentclass{article}
%%%%%%%%%%%%%%%%%%%%%%%%%%%%%%%%%%%%%%%%%%%%%%%%%%%%%%%%%%%%%%%%%%%%%%%%%%%%%%%

%%%%%%%%%%%%%%%%%%%%%%%%%%%%% Using Packages %%%%%%%%%%%%%%%%%%%%%%%%%%%%%%%%%%
\usepackage{geometry}
\usepackage{graphicx}
\usepackage{amssymb}
\usepackage{amsmath}
\usepackage{amsthm}
\usepackage{empheq}
\usepackage{mdframed}
\usepackage{booktabs}
\usepackage{lipsum}
\usepackage{graphicx}
\usepackage{color}
\usepackage{psfrag}
\usepackage{pgfplots}
\usepackage{bm}
%%%%%%%%%%%%%%%%%%%%%%%%%%%%%%%%%%%%%%%%%%%%%%%%%%%%%%%%%%%%%%%%%%%%%%%%%%%%%%%

% Other Settings

%%%%%%%%%%%%%%%%%%%%%%%%%% Page Setting %%%%%%%%%%%%%%%%%%%%%%%%%%%%%%%%%%%%%%%
\geometry{a4paper}

%%%%%%%%%%%%%%%%%%%%%%%%%% Define some useful colors %%%%%%%%%%%%%%%%%%%%%%%%%%
\definecolor{ocre}{RGB}{243,102,25}
\definecolor{mygray}{RGB}{243,243,244}
\definecolor{deepGreen}{RGB}{26,111,0}
\definecolor{shallowGreen}{RGB}{235,255,255}
\definecolor{deepBlue}{RGB}{61,124,222}
\definecolor{shallowBlue}{RGB}{235,249,255}
%%%%%%%%%%%%%%%%%%%%%%%%%%%%%%%%%%%%%%%%%%%%%%%%%%%%%%%%%%%%%%%%%%%%%%%%%%%%%%%

%%%%%%%%%%%%%%%%%%%%%%%%%% Define an orangebox command %%%%%%%%%%%%%%%%%%%%%%%%
\newcommand\orangebox[1]{\fcolorbox{ocre}{mygray}{\hspace{1em}#1\hspace{1em}}}
%%%%%%%%%%%%%%%%%%%%%%%%%%%%%%%%%%%%%%%%%%%%%%%%%%%%%%%%%%%%%%%%%%%%%%%%%%%%%%%

%%%%%%%%%%%%%%%%%%%%%%%%%%%% English Environments %%%%%%%%%%%%%%%%%%%%%%%%%%%%%
\newtheoremstyle{mytheoremstyle}{3pt}{3pt}{\normalfont}{0cm}{\rmfamily\bfseries}{}{1em}{{\color{black}\thmname{#1}~\thmnumber{#2}}\thmnote{\,--\,#3}}
\newtheoremstyle{myproblemstyle}{3pt}{3pt}{\normalfont}{0cm}{\rmfamily\bfseries}{}{1em}{{\color{black}\thmname{#1}~\thmnumber{#2}}\thmnote{\,--\,#3}}
\theoremstyle{mytheoremstyle}
\newmdtheoremenv[linewidth=1pt,backgroundcolor=shallowGreen,linecolor=deepGreen,leftmargin=0pt,innerleftmargin=20pt,innerrightmargin=20pt,]{theorem}{Theorem}[section]
\theoremstyle{mytheoremstyle}
\newmdtheoremenv[linewidth=1pt,backgroundcolor=shallowBlue,linecolor=deepBlue,leftmargin=0pt,innerleftmargin=20pt,innerrightmargin=20pt,]{definition}{Definition}[section]
\theoremstyle{myproblemstyle}
\newmdtheoremenv[linecolor=black,leftmargin=0pt,innerleftmargin=10pt,innerrightmargin=10pt,]{problem}{Problem}[section]
%%%%%%%%%%%%%%%%%%%%%%%%%%%%%%%%%%%%%%%%%%%%%%%%%%%%%%%%%%%%%%%%%%%%%%%%%%%%%%%

%%%%%%%%%%%%%%%%%%%%%%%%%%%%%%% Plotting Settings %%%%%%%%%%%%%%%%%%%%%%%%%%%%%
\usepgfplotslibrary{colorbrewer}
\pgfplotsset{width=8cm,compat=1.9}
%%%%%%%%%%%%%%%%%%%%%%%%%%%%%%%%%%%%%%%%%%%%%%%%%%%%%%%%%%%%%%%%%%%%%%%%%%%%%%%

%%%%%%%%%%%%%%%%%%%%%%%%%%%%%%% Title & Author %%%%%%%%%%%%%%%%%%%%%%%%%%%%%%%%
\title{Colored Cube}
\author{Zarak}
%%%%%%%%%%%%%%%%%%%%%%%%%%%%%%%%%%%%%%%%%%%%%%%%%%%%%%%%%%%%%%%%%%%%%%%%%%%%%%%

\begin{document}
    \maketitle

    Let \( A \) be the event that the face pointing down is painted. Let \( B
    \) be the event that the five visible faces are not painted. We need to
    find \( P(A \mid B) \) - the probability that the face pointing down is
    painted given that the five faces that are visible are not painted.

    According to \textbf{Bayes' theorem}, we have

    \begin{definition}[Bayes' Theorem]
      
      \begin{displaymath}
      P(A \mid B) = \frac{P(B \mid A) \cdot P(A)}{P(B)},
      \end{displaymath}
      
    \end{definition}
    

    where \( P(A) \) is the prior probability that the face pointing down is
    painted, \( P(B \mid A) \) is the likelihood, i.e. the probability that the
    five faces that are visible are not painted given that the face pointing
    down is painted, and \( P(B) \) is simply the probability that the five
    visible faces are not painted.

    First, we compute \( P(A) \). We have 
    \begin{align*}
      P(A) &= \frac{\text{\# of painted faces}}{\text{\# of total faces}} \\
           &= \frac{1 \times 6 + 2 \times 12 + 3 \times 8}{27 \times 6} \\
           &= \frac{54}{162} \\
           &= \frac{1}{3}.
    \end{align*}

    We now compute \( P(B \mid A) \). For a cube with 
    \begin{enumerate}
      \item one painted face, there is a \( 1 \) in \( 6 \) chance that the
        painted face is pointind down;
      \item two painted faces, there is a \( 2 \) in \( 6 \) chance;
      \item three painted faces, there is a \( 3 \) in \( 6 \) chance.
    \end{enumerate}

    We need to consider that when a painted face points down, that the other
    five visible faces are not painted; this can only occur in the case of one painted
    face. Thus,

    \begin{align*}
      P(B \mid A) &= \frac{(1\times 6) \times 1 + (2 \times 12) \times 0 + (3
      \times 8) \times  0}{1\times 6 + 2\times 12 + 3\times 8} \\
        &= \frac{6}{54} \\
        &= \frac{1}{9}.
    \end{align*}

    Next, we compute \( P (B) \), the probability that the visible faces are
    not painted. We consider each type of cube:

    \begin{enumerate}
      \item \( 1 \) cube with no painted faces: when this cube is rolled, the
        probability that visible faces are not painted is 1.
      \item cube with \( 1 \) painted face: there is a \( \frac{1}{6} \) chance
        that one of these cubes has its face pointing down.
      \item the remaining two types of cubes have probability \( 0 \) because
        at least one of the non-downward pointing faces would be visible.
    \end{enumerate}

    \begin{align*}
      P(B) = \frac{1}{27} \cdot 1 + \frac{6}{27} \cdot \frac{1}{6} +
      \frac{12}{27}\cdot 0 + \frac{8}{27} \cdot 0 = \frac{2}{27}.
    \end{align*}

    Therefore, our final result is
    \begin{align*}
      P(A \mid B) &= \frac{P(B \mid A) \cdot P(A)}{P(B)} \\
                  &= \frac{\frac{1}{9} \cdot \frac{1}{3}}{\frac{2}{27}}  \\
                  &= \frac{1}{2}.
    \end{align*}
\end{document}
